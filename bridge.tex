\documentclass[a5paper]{article}
\usepackage{fontspec,logbox}
\usepackage{xunicode}
\setmainfont{Linux Libertine O}
\usepackage{etoolbox}
\usepackage[margin=2cm]{geometry}
\def\size{}
\listadd{\size}{Huge}
\listadd{\size}{huge}
\listadd{\size}{LARGE}
\listadd{\size}{Large}
\listadd{\size}{large}
\listadd{\size}{normalsize}
\listadd{\size}{small}
\listadd{\size}{footnotesize}
\listadd{\size}{scriptsize}
\listadd{\size}{tiny}
\newcommand{\g}[1]{\textbf{#1}}
\begin{document}

\newcommand{\boitemax}[1]{
	\renewcommand{\do}[1]{
		\setbox0=\vbox{\csuse{##1}#1}
		\ifdimcomp{\ht0+\dp0}{<}{\textheight}{\listbreak}{}
	}
	\dolistloop{\size}
	\unvbox0
}

	\boitemax{
	\begin{tabular}{p{0.3\textwidth}|p{0.3\textwidth}|p{0.3\textwidth}}
	\textbf{Principes de base} & \textbf{Interventions sur 1 couleur} & \textbf{Défausse} \\
	
	♥~♠  5ième SMI 				
	Meilleure ♣~♦  ; SMI (sans 4C/P.),>10DH & Michael' CB (sauf bicolore noir > 1♦) & 1ère italienne à la couleur Lévinthal à SA \\
	
	\textbf{Ouvertures fortes} & \textbf{Interventions sur 2 majeur} & \textbf{Appel} \\
	
	>2♣  ;  ctrl italiens 2♦ (≤2)  ; 2♥ ->3♦ (3:4) >2♦ : 2♠ (As ♥/♠)  ; 3♣ ♦ (As ♣ ♦)  ; CRM picasso (3♥ ouvreur après 2♥ rép. = 5C+4P) & 4♣/♦ : l'autre majeure + la mineure annoncée & direct ou de préférence selon nbr car \\
	
	\g{Réponses particulières sur couleur} & \g{Interventions sur 3♣/♠ 4♣} & \\
	
	2SA fitté (id. >interv.) ; 3SA fitté 4° (12-14) splinters  ; 3ème forcing  ; 4ème forcing & 
	X : 6T ou 6K ou 5T/K + 4C  ; 2 ♣ landy
	2 ♦/♥ = texas  ; 2♠ = 4P+5T/K ; 2SA : 5T+5K
	3♣ à 3♠ : naturel barrage & \\
	
	\g{Redemandes spéciales 18-19 H régulier} & \g{Défenses sur interventions >1 SA}\\
	
	> 1♣/♦  [1♥/♠] : 3 SA fitté 18-19 H, rég. 
	> 1♣/♦  [1♥]  2SA ; possible 4P ou 3C
	autre mineure : interrog majeure
	& 
	
	Rubensohl  ; surcontre : texas impossible
	
	\\
	
	\g{Réponses particulières sur 1 SA} & \g{Défenses sur interventions > 1 couleur} & \\
	
	4♦ = 5p+5C 4 ♣ BW à voir après texas mineur : conv. Singleton après stayman : 3 autre M fitté fort & 2SA et 3SA Truscott		\\
	
	\g{Réponses particulières sur 1 SA} & \g{Réveil >1 SA} \\
	
	4♦ = 5p+5C 4 ♣ BW à voir après texas mineur : conv. Singleton  après stayman : 3 autre M fitté fort & 2♣ : appel majeures 4/4 ; X à voir & \\
	\g{Après Passe sur ouverture 1♥ /♠} & \g{Réveil >1 Couleur} \\
	
	2♣ drury  ; 2SA singl., saut fitté nouv coul. & X oblig. ≥ ouv  ; 2♦ bic. maj ; 2SA 17-19 1SA : 9-12   ; & \\
	
	\g{Blackwood} & \g{Réveil > après passe et deux couleurs} \\
	
	5 clefs étendu & 1 SA = 6/4							 \\
	
	\g{Stayman et texas} \\
	après 1 SA, 1SA d'Int ; 2SA, 2SA>2♣/♦  Pupett > 2SA, (et 2/3 SA >2♣/♦) \\
	
	\g{Après redemande 1 SA (1X/1Y/1SA)} \\
	 
	ping-pong \\
	\end{tabular}
}




\end{document}