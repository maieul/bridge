% Tu ne touche à rien de ce qu'il y a en dessous (toujours être réglé sur XeLaTeX)
\documentclass[twoside,a4paper]{article}
\usepackage{fontspec,logbox,multicol,xcolor}
\usepackage{xunicode}
\setmainfont{Linux Libertine O}
\usepackage{etoolbox}
\usepackage[headsep=0.1cm,inner=1cm,tmargin=0.6cm,outer=1cm,bmargin=0.1cm]{geometry}	%%% attention l'entete se trouve "dans" la marge
\setlength{\parskip}{3ex}										%%% le saut avant les débuts de paragraphe 1ex = la hauteur d'un x minuscule
\renewcommand{\columnseprulecolor}{\color{black}}				%%% couleur du séparateur de colon (en fait je ne sais pas pq, mais un peu plus clair)
\setlength{\columnseprule}{0.3pt}								%%% taille du séparateur de column
\parindent=0pt													%%% taille de l'indentation de la première ligne de ¶

\newcommand{\g}[1]{\textbf{#1}}


\newcommand{\boitemax}[1]{%
  \csdimdef{taille}{20pt}%											%%% la taille maximale de police (en pt(
  \setbox0=\vbox{\fontsize{\taille}{\taille}\selectfont #1}%
  \csdimdef{totale}{\ht0+\dp0}%
  \loop%
  \ifdim \totale> \textheight%
	\setbox0=\vbox{\fontsize{\taille}{\taille}\selectfont #1}%
	\csdimdef{taille}{\taille-0.1pt}%
	\csdimdef{totale}{\ht0+\dp0}%
	\GenericWarning{}{Taille de police : \taille}%
  \repeat%
  \unvbox0%
}
\let\exp\textsuperscript
\usepackage{fancyhdr}
\pagestyle{fancy}
\fancyfoot{}

\begin{document}
\fancyfoot{}
\fancyhead{}
% A partir d'ici tu peux toucher
\fancyhead[CE]{LIPIETZ Hélène 792 988  / ROUQUETTE Rémi 792 996}
\fancyhead[CO]{LIPIETZ Hélène 792 988  /  ROUQUETTE Rémi 792 996}

	\boitemax{					 % Entre les accolades de \boitemax{}, on dit à LaTeX d'essayer de ne pas dépasser la hauteur d'une page
	\g{Principes de base}\newline % Un \newline indique un changement de ligne (≠ du changement de ¶)
	♥~♠  5\exp{ème} ; Meilleure ♣~♦ souple  ; SMI (sans 4C/P.),>10DH		
	
	\g{Ouvertures fortes}\newline		 % \g{texte en gras}
	>2♣  ;  ctrl italiens 2♦ (≤2)  ; 2♥ ->3♦ (3/4)\newline
	 >2♦ : 2♠ (As ♥/♠)  ; 3♣ ♦ (As ♣ ♦)  ; CRM ; Picasso (3♠ ouvreur > rép. 2♥  = 5C+4P)\newline
	> 2SA répondant :  stayman virtuel (3♣ = jeu irrégulier)
	
	\g{Réponses particulières sur couleur} \newline
	2SA fitté (4 cartes >interv.) ; 3SA fitté 4° (12-14 H) ; splinters  ; 3ème forcing  ; 4ème forcing
	
	
		\g{Réponses particulières sur 1 SA} \newline
	4♦ = 5p+5C moyen ; 4 ♣ id. + ambition chelem ; \newline
	> texas mineur : conv. Singleton\newline ; 
	> stayman : 3 autre Majeure = fitté fort ; \newline
	> Stayman majeures croisées si 4/5 cartes ; texas 4♣/♦ si 6 cartes
	
	\g{ Enchères spéciales > redemande 2SA à saut 18-19 H possible 4P ou 3C } \newline
	> 1♣/♦  [1♥/♠]  2SA ; \newline
	autre mineure : interrog majeure; réponses : naturel + soutien fausse mineure = 3/4 majeur
		 
	 \g{Après  ouverture 1♥ /♠ en troisième ou quatrième} \newline
	2♣ drury  ; 2SA singl. ; enchères de rencontre (saut fitté dans nouvelle coul + 11DH) \newline
	2♦ fitté faible (5-7 DH) ; 2♥/♠ fitté fort (8-10 DH)
	
	\g{Enchères de Chelem}\newline
	Blackwood : 5 clefs étendu, 30-41 ; petit et grand Joséphine
	
	\g{Stayman et texas}\newline
		> 1 SA ; 1SA d'Intervention ou de réveil  \newline
		> 2SA ; 2SA d'interv. ou de réveil ; 2SA>2♣/♦ \newline
	\g{Stayman pupett}\newline
  	 > 2SA, (et 2/3 SA >2♣/♦) 
		
	\g{Redemande fitté à 3SA (1X/1Y/1SA)}\newline 
		> 1♣/♦  [1♥/♠] : 3 SA fitté 4 cartes 18-19 H, rég. 
 
	\g{Après redemande 1 SA (1X/1Y/1SA)}\newline 
	ping-pong; 2♦ oblig. sauf 2♥ /♠ fitté faible; 2♥ si singl. P et 4C
	}
	
	
	\boitemax{
	\begin{multicols}{2}
	 \g{Interventions sur 1 couleur}  \newline % Le  indique le changement de colonne, le // le changement de ligne
	Michael' CB  précisé (sauf bicolore noir > 1♦)   
	
	
	\g{Interventions sur 2 majeur}   \newline
	4♣/♦ : l'autre majeure + la mineure annoncée \newline  
	Cue-bid : bicolore mineur ; 4SA très beau bicolore mineur
	
	\g{Interventions sur 3♣/3♦/ 4♣} \newline
	4♦ = bicolore majeur 
	
	
	\g{Interventions sur 2,3,4♥/♠/} \newline
	4SA = bicolore mineur (sauf > 4♠, tous bic.)


	\g{Interventions sur 1 SA}  \newline
	X : 6T ou 6K ou 5T/K + 4C  ; 2 ♣ landy\newline
	2 ♦/♥ = texas  ; 2♠ = 4P+5T/K ; 2SA : 5T+5K \newline;
	3♣ à 3♠ : naturel barrage \newline
	surcontre SOS \newline
	––––––––––––––––––––––––––––
	
	 \g{Défenses sur interventions >1 SA}\newline
	Rubensohl  ; surcontre ; texas impossible singl. et autre Maj.
	
	

	 \g{Défenses sur interventions > 1 couleur}  \newline
	2SA et 3SA Truscott généralisés
	 
	 		
	 
	 \g{Défenses sur intervention par 1SA}  \newline
	Landik et texas (2♣ promet 2 majeures si non fitté ds mineure, 1 majeure si fitté) \newline
	   			
		 \vfill\columnbreak% &% la deuxieme colonne est en dessous

 	\g{Réveil >1 SA} \newline
	2♣ à 2♠ naturel \newline
	X 2 maj. 4ème ou + [réponses naturelles]
	 
	 
	 \g{Réveil >1 ♣/♦} \newline
	 X oblig. ≥ ouv  ; CB bic. maj ; 2SA 17-19 ; 1SA : 9-12   ; 
	 
	 

	  \g{Réveil >1 ♥/♠} \newline
	 X oblig. ≥ ouv  ; CB autre Maj + mineure indéterminée. ; 2SA 17-19 ; 1SA : 9-12   ; 


	\g{Réveil > après passe et deux couleurs}\newline
	 1 SA = 6 la moins chère/4	 la plus chère
	 
	 \rule{\linewidth}{1.5pt}
	 		 
	  \g{Défausse} \newline
	  1ère italienne à la couleur \newline
	  Lévinthal à SA
	  
	  
	  \g{Appel} \newline
	 direct ou de préférence selon nombre cartes
	 
	   						 
	 
	 
	 \g{Entame SA} \newline
	quatrième meilleure; honneur dans couleur quatrième de 4 grosses cartes ou cinquième de 3 grosses cartes, roi débloquant


	\g{Entame Couleur} \newline
	Pair impair\newline
	RA = doubleton de la couleur ou singleton autre couleur


	\g{Appel spéciaux} \newline
	appel de Smith

	
	\end{multicols}
	\vspace{-2\parskip}
}



\end{document}