% Tu ne touche à rien de ce qu'il y a en dessous (toujours être réglé sur XeLaTeX)
\documentclass[twoside,a5paper]{article}
\usepackage{fontspec,logbox,multicol,xcolor}
\usepackage{xunicode}
\setmainfont{Linux Libertine O}
\usepackage{etoolbox}
\usepackage[headsep=0.1cm,margin=0.1cm,tmargin=0.6cm]{geometry}	%%% attention l'entete se trouve "dans" la marge
\setlength{\parskip}{3ex}										%%% le saut avant les débuts de paragraphe 1ex = la hauteur d'un x minuscule
\renewcommand{\columnseprulecolor}{\color{black}}				%%% couleur du séparateur de colon (en fait je ne sais pas pq, mais un peu plus clair)
\setlength{\columnseprule}{0.3pt}								%%% taille du séparateur de column
\parindent=0pt													%%% taille de l'indentation de la première ligne de ¶


\def\size{}
\listadd{\size}{Huge}
\listadd{\size}{huge}
\listadd{\size}{LARGE}
\listadd{\size}{Large}
\listadd{\size}{large}
\listadd{\size}{normalsize}
\listadd{\size}{small}
\listadd{\size}{footnotesize}
\listadd{\size}{scriptsize}
\listadd{\size}{tiny}
\newcommand{\g}[1]{\textbf{#1}}
\newcommand{\boitemax}[1]{
	\renewcommand{\do}[1]{
		\setbox0=\vbox{\csuse{##1}#1}
		\ifdimcomp{\ht0+\dp0}{<}{\textheight}{\listbreak}{}
	}
	\dolistloop{\size}
	\unvbox0
}
\let\exp\textsuperscript
\usepackage{fancyhdr}
\pagestyle{fancy}
\fancyfoot{}

\begin{document}
\fancyfoot{}
\fancyhead{}
% A partir d'ici tu peux toucher
\fancyhead[CE]{LIPIETZ Hélène 792 988 - 4♠ / ROUQUETTE Rémi 792 996 - 4♠}
\fancyhead[CO]{LIPIETZ Hélène 792 988 - 4♠ /  ROUQUETTE Rémi 792 996 - 4♠}

	\boitemax{					 % Entre les accolades de \boitemax{}, on dit à LaTeX d'essayer de ne pas dépasser la hauteur d'une page
	\g{Principes de base}\newline % Un \newline indique un changement de ligne (≠ du changement de ¶)
	♥~♠  5\exp{ème} ; Meilleure ♣~♦  ; SMI (sans 4C/P.),>10DH		
	
	\g{Ouvertures fortes}\newline		 % \g{texte en gras}
	>2♣  ;  ctrl italiens 2♦ (≤2)  ; 2♥ ->3♦ (3:4)\newline
	 >2♦ : 2♠ (As ♥/♠)  ; 3♣ ♦ (As ♣ ♦)  ; CRM ; Picasso (3♥ ouvreur > rép. 2♥  = 5C+4P)
	
	\g{Réponses particulières sur couleur} \newline
	2SA fitté (id. >interv.) ; 3SA fitté 4° (12-14 H) ; splinters  ; 3ème forcing  ; 4ème forcing 
	
	\g{ Enchères répondant spéciales > redemande 2SA 18-19 H régulier} \newline
		> 1♣/♦  [1♥/♠]  2SA ; possible 4P ou 3C\newline
	> 2SA : autre mineure : interrog majeure; réponses : naturel + soutien fausse mineure = 4/3 majeur
	
		\g{Réponses particulières sur 1 SA} \newline
	4♦ = 5p+5C moyen ; 4 ♣ id. + ambition chelem ; 
	> texas mineur : conv. Singleton\newline ; > stayman : 3 autre Majeure = fitté fort ; 
	> Stayman majeures croisées si 4/5 cartes ; texas 4♣/♦ si 6 cartes
	 
	 \g{Après Passe sur ouverture 1♥ /♠} \newline
	2♣ drury  ; 2SA singl., enchères de rencontre (saut fitté dans nouvelle coul) \newline
	2♦ fitté faible (5-7 DH) ; 2♥/♠ fitté fort (8-10 DH)
	
	\g{Enchères de Chelem}\newline
	Blackwood : 5 clefs étendu, 30-41 ; Joséphine
	
	\g{Stayman et texas}\newline
		> 1 SA, 1SA d'Int ; 2SA, 2SA d'int. ; 2SA>2♣/♦ 
	\g{Stayman pupett}\newline
  	 > 2SA, (et 2/3 SA >2♣/♦) 
	
	\g{Après redemande 1 SA (1X/1Y/1SA)}\newline 
	ping-pong; 2♦ oblig. sauf 2♥ /♠ fitté faible; 2♥ si singl. P et 4C
	\newline
	}
	
	
	\boitemax{
	\begin{multicols}{2}
	 \g{Interventions sur 1 couleur}  \newline % Le  indique le changement de colonne, le // le changement de ligne
	Michael' CB (sauf bicolore noir > 1♦)   
	
	
	\g{Interventions sur 2 majeur}   \newline
	4♣/♦ : l'autre majeure + la mineure annoncée   
	
	
	\g{Interventions sur 3♣/3♦/ 4♣} \newline
	4♦ = bicolore majeur 
	
	
	\g{Interventions sur 2,3,4♥/♠/} \newline
	4SA = bicolore mineur 


	\g{Interventions sur 1 SA}  \newline
	X : 6T ou 6K ou 5T/K + 4C  ; 2 ♣ landy
	2 ♦/♥ = texas  ; 2♠ = 4P+5T/K ; 2SA : 5T+5K \newline;
	3♣ à 3♠ : naturel barrage 
	
	
	 \g{Défenses sur interventions >1 SA}\newline
	> 1♣/♦  [1♥/♠] : 3 SA fitté 18-19 H, rég. 
	Rubensohl  ; surcontre : texas impossible
	
	

	 \g{Défenses sur interventions > 1 couleur}  \newline
	2SA et 3SA Truscott
	 
	 		
	 
	 \g{Défenses sur intervention par 1SA}  \newline
	Landik et texas (2♣ promet 2 majeures si non fitté ds mineure, 1 majeure si fitté)
	 	
			
	 \g{Réveil >1 SA} \newline
	2♣ : appel majeures 4/4 ; X à voir 
	 
	 
	 \g{Réveil >1 ♣/♦} \newline
	 X oblig. ≥ ouv  ; CB bic. maj ; 2SA 17-19 1SA : 9-12   ; 
	 
	 
	  \g{Réveil >1 ♥/♠} \newline
	 X oblig. ≥ ouv  ; CB bic. min. ; 2SA 17-19 ; 1SA : 9-12   ; 


	\g{Réveil > après passe et deux couleurs}
	 1 SA = 6/4	
	 
	
	  						 
	 
	 \vfill\columnbreak% &% la deuxieme colonne est en dessous
	 
	  \g{Défausse} \newline
	  1ère italienne à la couleur 
	  Lévinthal à SA
	  
	  
	  \g{Appel} \newline
	 direct ou de préférence selon nbr cartes
	 
	 
	 
	 \g{Entame SA} \newline
	quatrième meilleure; honneur dans couleur quatrième de 4 grosses cartes ou cinquième de 3 grosses cartes, roi débloquant


	\g{Entame Couleur} \newline
	RA = doubleton de la couleur ou singleton autre couleur


	\g{appel spéciaux} \newline
	appel de Smith

	
	\end{multicols}
}



\end{document}