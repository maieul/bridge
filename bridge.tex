% Tu ne touche à rien de ce qu'il y a en dessous (toujours être réglé sur XeLaTeX)
\documentclass[twoside,a5paper]{article}
\usepackage{fontspec,logbox}
\usepackage{xunicode}
\setmainfont{Linux Libertine O}
\usepackage{etoolbox}
\usepackage[headsep=0.4cm,margin=1.0cm,tmargin=2cm]{geometry}
\def\size{}
\listadd{\size}{Huge}
\listadd{\size}{huge}
\listadd{\size}{LARGE}
\listadd{\size}{Large}
\listadd{\size}{large}
\listadd{\size}{normalsize}
\listadd{\size}{small}
\listadd{\size}{footnotesize}
\listadd{\size}{scriptsize}
\listadd{\size}{tiny}
\newcommand{\g}[1]{\textbf{#1}}
\newcommand{\boitemax}[1]{
	\renewcommand{\do}[1]{
		\setbox0=\vbox{\csuse{##1}#1}
		\ifdimcomp{\ht0+\dp0}{<}{\textheight}{\listbreak}{}
	}
	\dolistloop{\size}
	\unvbox0
}
\setlength{\parskip}{1ex}
\let\exp\textsuperscript
\usepackage{fancyhdr}
\pagestyle{fancy}
\fancyfoot{}
\parindent=0pt
\begin{document}
\fancyfoot{}
\fancyhead{}
% A partir d'ici tu peux toucher
\fancyhead[CE]{LIPIETZ Hélène 792 988 - 4♠ / ROUQUETTE Rémi 792 996 - 4♠}
\fancyhead[CO]{LIPIETZ Hélène 792 988 - 4♠ /  ROUQUETTE Rémi 792 996 - 4♠}

	\boitemax{					 % Entre les accolades de \boitemax{}, on dit à LaTeX d'essayer de ne pas dépasser la hauteur d'une page
	\g{Principes de base}\newline % Un \newline indique un changement de ligne (≠ du changement de ¶)
	♥~♠  5\exp{ème} SMI 				\newline
	Meilleure ♣~♦  ; SMI (sans 4C/P.),>10DH
	
	\g{Ouvertures fortes}\newline		 % \g{texte en gras}
	>2♣  ;  ctrl italiens 2♦ (≤2)  ; 2♥ ->3♦ (3:4) >2♦ : 2♠ (As ♥/♠)  ; 3♣ ♦ (As ♣ ♦)  ; CRM picasso (3♥ ouvreur après 2♥ rép. = 5C+4P)
	
	\g{Réponses particulières sur couleur} \newline
	2SA fitté (id. >interv.) ; 3SA fitté 4° (12-14) splinters  ; 3ème forcing  ; 4ème forcing 
	
	\g{Redemandes spéciales 18-19 H régulier} \newline
		> 1♣/♦  [1♥]  2SA ; possible 4P ou 3C
	autre mineure : interrog majeure
	
	
	\g{Réponses particulières sur 1 SA} \newline
	4♦ = 5p+5C 4 ♣ BW à voir après texas mineur : conv. Singleton après stayman : 3 autre M fitté fort 
	
	\g{Réponses particulières sur 1 SA} \newline
	\fbox{4♦} = 5p+5C 4 ♣ BW à voir après texas mineur : conv. Singleton  après stayman : 3 autre M fitté fort 
	
	\g{Après Passe sur ouverture 1♥ /♠} \newline
	2♣ drury  ; 2SA singl., saut fitté nouv coul. 
	
	\g{Blackwood}\newline
	5 clefs étendu
	
	\g{Stayman et texas}\newline
		après 1 SA, 1SA d'Int ; 2SA, 2SA>2♣/♦  Pupett > 2SA, (et 2/3 SA >2♣/♦) 
	
	\g{Après redemande 1 SA (1X/1Y/1SA)}\newline 
	ping-pong
	\newline
	}
	
	\boitemax{
	
	\begin{tabular}{p{0.48\textwidth}|p{0.48\textwidth}}		% Pour avoir deux colonnes (en théorie tu n'a pas à touché)
	 \g{Interventions sur 1 couleur}  \newline % Le  indique le changement de colonne, le // le changement de ligne
	Michael' CB (sauf bicolore noir > 1♦)   
	
	\g{Interventions sur 2 majeur}   \newline
	4♣/♦ : l'autre majeure + la mineure annoncée   
	
	\g{Interventions sur 3♣/♠ 4♣}  \newline
	X : 6T ou 6K ou 5T/K + 4C  ; 2 ♣ landy
	2 ♦/♥ = texas  ; 2♠ = 4P+5T/K ; 2SA : 5T+5K
	3♣ à 3♠ : naturel barrage 
	
	 \g{Défenses sur interventions >1 SA}\newline
	> 1♣/♦  [1♥/♠] : 3 SA fitté 18-19 H, rég. 
	Rubensohl  ; surcontre : texas impossible
	
	
	 \g{Défenses sur interventions > 1 couleur}  \newline
	2SA et 3SA Truscott
	 		
	 
	 \g{Réveil >1 SA} \newline
	2♣ : appel majeures 4/4 ; X à voir 
	 
	 \g{Réveil >1 Couleur} \newline
	 X oblig. ≥ ouv  ; 2♦ bic. maj ; 2SA 17-19 1SA : 9-12   ; 
	 
	 \g{Réveil > après passe et deux couleurs}
	 1 SA = 6/4	
	 
	
	  						 
	 
	  &% la deuxieme colonne est en dessous
	 
	  \g{Défausse} \newline
	  1ère italienne à la couleur Lévinthal à SA
	  
	  \g{Appel} \newline
	 direct ou de préférence selon nbr car
	 \\
	

	\end{tabular}							% Fin du tableau

}



\end{document}